% Options for packages loaded elsewhere
% Options for packages loaded elsewhere
\PassOptionsToPackage{unicode}{hyperref}
\PassOptionsToPackage{hyphens}{url}
\PassOptionsToPackage{dvipsnames,svgnames,x11names}{xcolor}
%
\documentclass[
  letterpaper,
  DIV=11,
  numbers=noendperiod]{scrartcl}
\usepackage{xcolor}
\usepackage{amsmath,amssymb}
\setcounter{secnumdepth}{5}
\usepackage{iftex}
\ifPDFTeX
  \usepackage[T1]{fontenc}
  \usepackage[utf8]{inputenc}
  \usepackage{textcomp} % provide euro and other symbols
\else % if luatex or xetex
  \usepackage{unicode-math} % this also loads fontspec
  \defaultfontfeatures{Scale=MatchLowercase}
  \defaultfontfeatures[\rmfamily]{Ligatures=TeX,Scale=1}
\fi
\usepackage{lmodern}
\ifPDFTeX\else
  % xetex/luatex font selection
\fi
% Use upquote if available, for straight quotes in verbatim environments
\IfFileExists{upquote.sty}{\usepackage{upquote}}{}
\IfFileExists{microtype.sty}{% use microtype if available
  \usepackage[]{microtype}
  \UseMicrotypeSet[protrusion]{basicmath} % disable protrusion for tt fonts
}{}
\makeatletter
\@ifundefined{KOMAClassName}{% if non-KOMA class
  \IfFileExists{parskip.sty}{%
    \usepackage{parskip}
  }{% else
    \setlength{\parindent}{0pt}
    \setlength{\parskip}{6pt plus 2pt minus 1pt}}
}{% if KOMA class
  \KOMAoptions{parskip=half}}
\makeatother
% Make \paragraph and \subparagraph free-standing
\makeatletter
\ifx\paragraph\undefined\else
  \let\oldparagraph\paragraph
  \renewcommand{\paragraph}{
    \@ifstar
      \xxxParagraphStar
      \xxxParagraphNoStar
  }
  \newcommand{\xxxParagraphStar}[1]{\oldparagraph*{#1}\mbox{}}
  \newcommand{\xxxParagraphNoStar}[1]{\oldparagraph{#1}\mbox{}}
\fi
\ifx\subparagraph\undefined\else
  \let\oldsubparagraph\subparagraph
  \renewcommand{\subparagraph}{
    \@ifstar
      \xxxSubParagraphStar
      \xxxSubParagraphNoStar
  }
  \newcommand{\xxxSubParagraphStar}[1]{\oldsubparagraph*{#1}\mbox{}}
  \newcommand{\xxxSubParagraphNoStar}[1]{\oldsubparagraph{#1}\mbox{}}
\fi
\makeatother


\usepackage{longtable,booktabs,array}
\usepackage{calc} % for calculating minipage widths
% Correct order of tables after \paragraph or \subparagraph
\usepackage{etoolbox}
\makeatletter
\patchcmd\longtable{\par}{\if@noskipsec\mbox{}\fi\par}{}{}
\makeatother
% Allow footnotes in longtable head/foot
\IfFileExists{footnotehyper.sty}{\usepackage{footnotehyper}}{\usepackage{footnote}}
\makesavenoteenv{longtable}
\usepackage{graphicx}
\makeatletter
\newsavebox\pandoc@box
\newcommand*\pandocbounded[1]{% scales image to fit in text height/width
  \sbox\pandoc@box{#1}%
  \Gscale@div\@tempa{\textheight}{\dimexpr\ht\pandoc@box+\dp\pandoc@box\relax}%
  \Gscale@div\@tempb{\linewidth}{\wd\pandoc@box}%
  \ifdim\@tempb\p@<\@tempa\p@\let\@tempa\@tempb\fi% select the smaller of both
  \ifdim\@tempa\p@<\p@\scalebox{\@tempa}{\usebox\pandoc@box}%
  \else\usebox{\pandoc@box}%
  \fi%
}
% Set default figure placement to htbp
\def\fps@figure{htbp}
\makeatother


% definitions for citeproc citations
\NewDocumentCommand\citeproctext{}{}
\NewDocumentCommand\citeproc{mm}{%
  \begingroup\def\citeproctext{#2}\cite{#1}\endgroup}
\makeatletter
 % allow citations to break across lines
 \let\@cite@ofmt\@firstofone
 % avoid brackets around text for \cite:
 \def\@biblabel#1{}
 \def\@cite#1#2{{#1\if@tempswa , #2\fi}}
\makeatother
\newlength{\cslhangindent}
\setlength{\cslhangindent}{1.5em}
\newlength{\csllabelwidth}
\setlength{\csllabelwidth}{3em}
\newenvironment{CSLReferences}[2] % #1 hanging-indent, #2 entry-spacing
 {\begin{list}{}{%
  \setlength{\itemindent}{0pt}
  \setlength{\leftmargin}{0pt}
  \setlength{\parsep}{0pt}
  % turn on hanging indent if param 1 is 1
  \ifodd #1
   \setlength{\leftmargin}{\cslhangindent}
   \setlength{\itemindent}{-1\cslhangindent}
  \fi
  % set entry spacing
  \setlength{\itemsep}{#2\baselineskip}}}
 {\end{list}}
\usepackage{calc}
\newcommand{\CSLBlock}[1]{\hfill\break\parbox[t]{\linewidth}{\strut\ignorespaces#1\strut}}
\newcommand{\CSLLeftMargin}[1]{\parbox[t]{\csllabelwidth}{\strut#1\strut}}
\newcommand{\CSLRightInline}[1]{\parbox[t]{\linewidth - \csllabelwidth}{\strut#1\strut}}
\newcommand{\CSLIndent}[1]{\hspace{\cslhangindent}#1}



\setlength{\emergencystretch}{3em} % prevent overfull lines

\providecommand{\tightlist}{%
  \setlength{\itemsep}{0pt}\setlength{\parskip}{0pt}}



 


\KOMAoption{captions}{tableheading}
\makeatletter
\@ifpackageloaded{caption}{}{\usepackage{caption}}
\AtBeginDocument{%
\ifdefined\contentsname
  \renewcommand*\contentsname{Table of contents}
\else
  \newcommand\contentsname{Table of contents}
\fi
\ifdefined\listfigurename
  \renewcommand*\listfigurename{List of Figures}
\else
  \newcommand\listfigurename{List of Figures}
\fi
\ifdefined\listtablename
  \renewcommand*\listtablename{List of Tables}
\else
  \newcommand\listtablename{List of Tables}
\fi
\ifdefined\figurename
  \renewcommand*\figurename{Figure}
\else
  \newcommand\figurename{Figure}
\fi
\ifdefined\tablename
  \renewcommand*\tablename{Table}
\else
  \newcommand\tablename{Table}
\fi
}
\@ifpackageloaded{float}{}{\usepackage{float}}
\floatstyle{ruled}
\@ifundefined{c@chapter}{\newfloat{codelisting}{h}{lop}}{\newfloat{codelisting}{h}{lop}[chapter]}
\floatname{codelisting}{Listing}
\newcommand*\listoflistings{\listof{codelisting}{List of Listings}}
\makeatother
\makeatletter
\makeatother
\makeatletter
\@ifpackageloaded{caption}{}{\usepackage{caption}}
\@ifpackageloaded{subcaption}{}{\usepackage{subcaption}}
\makeatother
\usepackage{bookmark}
\IfFileExists{xurl.sty}{\usepackage{xurl}}{} % add URL line breaks if available
\urlstyle{same}
\hypersetup{
  pdftitle={Analysing decisions in data analysis},
  colorlinks=true,
  linkcolor={blue},
  filecolor={Maroon},
  citecolor={Blue},
  urlcolor={Blue},
  pdfcreator={LaTeX via pandoc}}


\title{Analysing decisions in data analysis}
\author{H. Sherry Zhang \and Roger D. Peng}
\date{}
\begin{document}
\maketitle


\section{Introduction}\label{introduction}

In this work, we design a tabular format to record the choices made by
analysts during data analysis. Using large language models, we
automatically extract these choices from a set of research papers
focused on specific topics, e.g.~air pollution modelling. This allows us
to analyze these choices as data -- tracking how they've changed over
time or query the possible methodologies used in similar studies. We
also introduce a workflow to cluster paper based on decision similarity,
using both the decisions themselves and the justifications authors
provide for their choices.

\section{Background}\label{background}

Data analysis as an complicated, iterative process to make sense
{[}ref{]} of the data collected. The iterative process of formulating
hypothesis Jun et al. (\citeproc{ref-jun2022}{2022}).

Choices are made at nearly every stage of data analysis, ranging from
variable pre-processing variables, variable and lag selection in model
formulation, to the specification of smoothing parameter during model
construction. These possible choices contribute to what Gelman and Loken
(\citeproc{ref-gelman2014}{2014}) describe as the ``garden of forking
paths''. These choices can introduce substantial variability in results,
which has been demonstrated in many-analyst experiments, where
independent teams analyzing the same dataset to answer a pre-defined
research question often arrive at markedly different conclusions. A
prominent example is Silberzahn et al.
(\citeproc{ref-silberzahn2018}{2018}) where researchers reported a wide
range of point estimates and 95\% confidence intervals for the effect of
soccer players' skin tone on the number of red cards awarded by referees
(odds ratio from 0.89 to 2.93). Similar findings have emerged in other
domains, including structural equation modeling
(\citeproc{ref-sarstedt2024}{Sarstedt et al. 2024}), applied
microeconomics (\citeproc{ref-huntington-klein2021}{Huntington-Klein et
al. 2021}), neuroimaging
(\citeproc{ref-botvinik-nezer2020}{Botvinik-Nezer et al. 2020}), and
ecology and evolutionary biology (\citeproc{ref-gould2025}{Gould et al.
2025}).

Given this nature that a wide range of choices can be justified, there
has been the long discussion of p-hacking and other misuse of statistics
in practical data analysis in publication. {[}more on p-hacking and
reseracher's degree of freedom{]}. Some proposed guidelines and
recommendation to overcome this through pre-registration.

\begin{itemize}
\tightlist
\item
  pre-registration
\item
  (\citeproc{ref-wicherts2016degrees}{\textbf{wicherts2016degrees?}})
  provides a checklist of researcher degrees of freedom to combat the
  ``garden of forking paths'' problem
\item
  false-positive psychology suggests 6 guidelines for reporting data
  analysis to prevent false-positive rate
\end{itemize}

Another line of work focuses on developing software tools to support
analysts in making more informed decisions. For example, the
\texttt{Tisane} package (\citeproc{ref-jun2022}{Jun et al. 2022})
integrates conceptual ideas, such as DAGs, and modelling structure
(group/ cluster/ hierarchical structure), to assist junior researchers
in specifying GLM and GLMM model. The \texttt{DeclareDesign} package
(\citeproc{ref-blair2019}{Blair et al. 2019}) introduces the MIDA
framework for researchers to declare, diagnose, and redesign their
analyses to produce a distribution of the statistic of interest. This
approach has been applied in randomized controlled trial
(\citeproc{ref-bishop2024}{Bishop and Hulme 2024}) . The
\texttt{multiverse} package facilitates the specification and execution
of multiple parallel choices for sensitivity analysis, allowing
researchers to systematically explore how different choices affect
results and to report the range of plausible outcomes that arise from
alternative analytic paths.

\begin{itemize}
\item
  Study decisions in data analysis:

  \begin{itemize}
  \tightlist
  \item
    an interview/ participatory method:
    (\citeproc{ref-simson}{\textbf{simson?}}), many others
  \item
    (automated) visualization of decisions made in data analysis:

    \begin{itemize}
    \tightlist
    \item
      datamation (\citeproc{ref-pu2023}{Pu and Kay 2023}),
    \item
      (\citeproc{ref-liu2020paths}{\textbf{liu2020paths?}}) uses
      Analytic Decision Graphs (ADG) to represent high-level decision
      process in data analysis.
    \item
      \href{https://arxiv.org/pdf/2306.07760}{Urania: Visualizing Data
      Analysis Pipelines for Natural Language-Based Data Exploration}
    \end{itemize}
  \end{itemize}
\end{itemize}

\section{Construct decision
databases}\label{construct-decision-databases}

\subsection{Recording decisions in data
analysis}\label{recording-decisions-in-data-analysis}

\begin{itemize}
\tightlist
\item
  give example from extracting decision from sentences of a paper
\item
  adapt from the tidy data principle
  (\citeproc{ref-tidydata}{\textbf{tidydata?}}), each row is a decision
  Wickham (\citeproc{ref-wickham2014}{2014})\\
\item
  some decisions are related to how the variable is estimated spatially
  and temporally
\item
  model level decisions on how the model is estimated spatially (for
  multi-site analyses) and/or temporally (different treatments for years
  or seasons)
\item
  sometimes the decisions are not explicitly stated in the paper (use
  AIC to choose the degree of freedom in a smoothing spline)
\item
  sometimes the reason is not explicitly stated (e.g., why 3 degree of
  freedom)
\end{itemize}

A hypothetical database of decisions may look as follows:

\begin{longtable}[]{@{}
  >{\raggedright\arraybackslash}p{(\linewidth - 16\tabcolsep) * \real{0.1111}}
  >{\raggedright\arraybackslash}p{(\linewidth - 16\tabcolsep) * \real{0.1111}}
  >{\raggedright\arraybackslash}p{(\linewidth - 16\tabcolsep) * \real{0.1111}}
  >{\raggedright\arraybackslash}p{(\linewidth - 16\tabcolsep) * \real{0.1111}}
  >{\raggedright\arraybackslash}p{(\linewidth - 16\tabcolsep) * \real{0.1111}}
  >{\raggedright\arraybackslash}p{(\linewidth - 16\tabcolsep) * \real{0.1111}}
  >{\raggedright\arraybackslash}p{(\linewidth - 16\tabcolsep) * \real{0.1111}}
  >{\raggedright\arraybackslash}p{(\linewidth - 16\tabcolsep) * \real{0.1111}}
  >{\raggedright\arraybackslash}p{(\linewidth - 16\tabcolsep) * \real{0.1111}}@{}}
\toprule\noalign{}
\begin{minipage}[b]{\linewidth}\raggedright
Paper
\end{minipage} & \begin{minipage}[b]{\linewidth}\raggedright
ID
\end{minipage} & \begin{minipage}[b]{\linewidth}\raggedright
Model
\end{minipage} & \begin{minipage}[b]{\linewidth}\raggedright
variable
\end{minipage} & \begin{minipage}[b]{\linewidth}\raggedright
method
\end{minipage} & \begin{minipage}[b]{\linewidth}\raggedright
parameter
\end{minipage} & \begin{minipage}[b]{\linewidth}\raggedright
type
\end{minipage} & \begin{minipage}[b]{\linewidth}\raggedright
reason
\end{minipage} & \begin{minipage}[b]{\linewidth}\raggedright
decision
\end{minipage} \\
\midrule\noalign{}
\endhead
\bottomrule\noalign{}
\endlastfoot
ostro & 1 & Poisson regression & temperature & smoothing spline & degree
of freedom & parameter & NA & 3 degree of freedom \\
ostro & 2 & Poisson regression & temperature & smoothing spline & degree
of freedom & temporal & NA & 1-day lag \\
ostro & 3 & Poisson regression & relative humidity & LOESS & smoothing
parameter & parameter & to minimize Akaike's Information Criterion &
NA \\
ostro & 4 & Poisson regression & model & NA & NA & spatial & to account
for variation among cities & separate regression models fit in each
city \\
\end{longtable}

\subsection{Austomatic reading of literature with
LLM}\label{austomatic-reading-of-literature-with-llm}

\begin{itemize}
\tightlist
\item
  We use LLM to automatic read the paper through the \texttt{ellmer}
  package (\citeproc{ref-ellmer}{Wickham, Cheng, and Jacobs 2025}) and
  manually review the decision outputs. Both Antropic Claude and Google
  Gemini accept pdf inputs and we choose Claude. The prompt used to
  finetune the Claude LLM is available in the appendix.
\end{itemize}

\subsection{Review the LLM output}\label{review-the-llm-output}

(the shiny app)

\begin{itemize}
\tightlist
\item
  screenshot of the interface
\item
  The current application includes three actions:

  \begin{itemize}
  \tightlist
  \item
    modify a row
    (\texttt{dplyr::mutate(xxx\ =\ ifelse(CONDITION,\ "yyy"\ ,\ xxx))}),
  \item
    delete unrelated decisions (\texttt{dplyr::filter(!(CONDITION))}),
    and
  \item
    manually add a decision (\texttt{dplyr::bind\_rows()})
  \end{itemize}
\item
  All the actions will generate the corresponding codes.
\item
  The download button will download the modified decision database as a
  csv file
\end{itemize}

\subsubsection{Decision quality summary}\label{decision-quality-summary}

\section{Calculate paper similarity}\label{calculate-paper-similarity}

\begin{itemize}
\tightlist
\item
  pre-processing

  \begin{itemize}
  \tightlist
  \item
    standardize statistical methods its corresponding parameters (LOESS,
    smoothing spline, etc)
  \item
    group variables into broader categories: time, temperature,
    humidity, PM
  \end{itemize}
\item
  identify the most frequent analysis decisions across papers
\item
  retain only papers that report more than x such decisions
\item
  measure similarity between decisions and their justificaiton using NLP

  \begin{itemize}
  \tightlist
  \item
    word embedding with attention mechanism, instead of bag of word,
  \item
    specific NLP models (default to \texttt{bert-base-uncased}),
    aggregation methods from word to text
  \end{itemize}
\item
  compute paper similarity score for each paper pair by aggregating
  decision-level compoarisons

  \begin{itemize}
  \tightlist
  \item
    check/ report on the number of decisions compared in each paper pair
  \end{itemize}
\item
  similarity score can serve as the distance matrix to cluster papers by
  their similarity on decision choices
\end{itemize}

\section{Applications}\label{applications}

\subsection{Air pollution mortality
modelling}\label{air-pollution-mortality-modelling}

\begin{itemize}
\tightlist
\item
  look at for one type of decision (time) - what are the choices made by
  different papers
\item
  look at whether decisions changes across time
\item
  Visualize the decision database: apply clustering algorithm and
  visualize the database through \texttt{sigma.js}
\end{itemize}

\subsection{Species distribution
modelling}\label{species-distribution-modelling}

\section{Discussion}\label{discussion}

\begin{itemize}
\tightlist
\item
  Only prompting engineering is used to extract decisions from the
  literature. We expect that fine-tuning the model on statistical or
  domain-specific literature to yield more robust performance on the
  same document, though it would require substantially more training
  effort.
\end{itemize}

\section*{Reference}\label{reference}
\addcontentsline{toc}{section}{Reference}

\phantomsection\label{refs}
\begin{CSLReferences}{1}{0}
\bibitem[\citeproctext]{ref-bishop2024}
Bishop, Dorothy V. M., and Charles Hulme. 2024. {``When Alternative
Analyses of the Same Data Come to Different Conclusions: A Tutorial
Using DeclareDesign With a Worked Real-World Example.''} \emph{Advances
in Methods and Practices in Psychological Science} 7 (3):
25152459241267904. \url{https://doi.org/10.1177/25152459241267904}.

\bibitem[\citeproctext]{ref-blair2019}
Blair, Graeme, Jasper Cooper, Alexander Coppock, and Macartan Humphreys.
2019. {``Declaring and Diagnosing Research Designs.''} \emph{American
Political Science Review} 113 (3): 838--59.
\url{https://doi.org/10.1017/S0003055419000194}.

\bibitem[\citeproctext]{ref-botvinik-nezer2020}
Botvinik-Nezer, Rotem, Felix Holzmeister, Colin F. Camerer, Anna Dreber,
Juergen Huber, Magnus Johannesson, Michael Kirchler, et al. 2020.
{``Variability in the Analysis of a Single Neuroimaging Dataset by Many
Teams.''} \emph{Nature} 582 (7810): 84--88.
\url{https://doi.org/10.1038/s41586-020-2314-9}.

\bibitem[\citeproctext]{ref-gelman2014}
Gelman, Andrew, and Eric Loken. 2014. {``The Statistical Crisis in
Science.''} \emph{American Scientist} 102 (6): 460--65.
\url{https://www.proquest.com/docview/1616141998/abstract/5E050DCE82414037PQ/1}.

\bibitem[\citeproctext]{ref-gould2025}
Gould, Elliot, Hannah S. Fraser, Timothy H. Parker, Shinichi Nakagawa,
Simon C. Griffith, Peter A. Vesk, Fiona Fidler, et al. 2025. {``Same
Data, Different Analysts: Variation in Effect Sizes Due to Analytical
Decisions in Ecology and Evolutionary Biology.''} \emph{BMC Biology} 23
(1): 35. \url{https://doi.org/10.1186/s12915-024-02101-x}.

\bibitem[\citeproctext]{ref-huntington-klein2021}
Huntington-Klein, Nick, Andreu Arenas, Emily Beam, Marco Bertoni,
Jeffrey R. Bloem, Pralhad Burli, Naibin Chen, et al. 2021. {``The
Influence of Hidden Researcher Decisions in Applied Microeconomics.''}
\emph{Economic Inquiry} 59 (3): 944--60.
\url{https://doi.org/10.1111/ecin.12992}.

\bibitem[\citeproctext]{ref-jun2022}
Jun, Eunice, Melissa Birchfield, Nicole De Moura, Jeffrey Heer, and René
Just. 2022. {``Hypothesis Formalization: Empirical Findings, Software
Limitations, and Design Implications.''} \emph{ACM Transactions on
Computer-Human Interaction} 29 (1): 1--28.
\url{https://doi.org/10.1145/3476980}.

\bibitem[\citeproctext]{ref-pu2023}
Pu, Xiaoying, and Matthew Kay. 2023. {``CHI '23: CHI Conference on Human
Factors in Computing Systems.''} In, 1--22. Hamburg Germany: ACM.
\url{https://doi.org/10.1145/3544548.3580837}.

\bibitem[\citeproctext]{ref-sarstedt2024}
Sarstedt, Marko, Susanne J. Adler, Christian M. Ringle, Gyeongcheol Cho,
Adamantios Diamantopoulos, Heungsun Hwang, and Benjamin D. Liengaard.
2024. {``Same Model, Same Data, but Different Outcomes: Evaluating the
Impact of Method Choices in Structural Equation Modeling.''}
\emph{Journal of Product Innovation Management} 41 (6): 1100--1117.
\url{https://doi.org/10.1111/jpim.12738}.

\bibitem[\citeproctext]{ref-silberzahn2018}
Silberzahn, R., E. L. Uhlmann, D. P. Martin, P. Anselmi, F. Aust, E.
Awtrey, Š. Bahník, et al. 2018. {``Many Analysts, One Data Set: Making
Transparent How Variations in Analytic Choices Affect Results.''}
\emph{Advances in Methods and Practices in Psychological Science} 1 (3):
337--56. \url{https://doi.org/10.1177/2515245917747646}.

\bibitem[\citeproctext]{ref-wickham2014}
Wickham, Hadley. 2014. {``Tidy Data.''} \emph{Journal of Statistical
Software} 59 (September): 1--23.
\url{https://doi.org/10.18637/jss.v059.i10}.

\bibitem[\citeproctext]{ref-ellmer}
Wickham, Hadley, Joe Cheng, and Aaron Jacobs. 2025. \emph{Ellmer: Chat
with Large Language Models}.
\url{https://CRAN.R-project.org/package=ellmer}.

\end{CSLReferences}




\end{document}
